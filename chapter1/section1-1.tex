\subsection{Section 1: Deductive Reasoning and Logical Connectives}

\begin{questions}

\subsubsection{Exercise 1}

\question Analyze the logical forms of the following statements:
\begin{parts}
    \part  We’ll have either a reading assignment or homework problems, but we
won’t have both homework problems and a test.
\part You won’t go skiing, or you will and there won’t be any snow.
\part $\sqrt{7}\nleq 2$
\end{parts}

\begin{solution}
    \begin{enumerate}[label=(\alph*)]
        \item P:= We'll have reading assignment. \;\\
                Q:= We'll have homework problems. \\
                T:= We'll have test. \\
                $(P \vee Q) \wedge \neg(Q \wedge T)$
                
        \item P:= You go skiing \\
                Q:= There will be snow \\
                $\neg P \vee (P \wedge \neg Q)$
        \item $\sqrt{7} \nleq 2$ \\
                P:= $\sqrt{7} < 2$ \\
                Q:= $\sqrt{7} = 2$ \\
                $\neg (P \vee Q)$
    \end{enumerate}
\end{solution}

%====================================================================================================

\subsubsection{Exercise 2}
\question Analyze the logical forms of the following statements:

\begin{parts}
    \part Either John and Bill are both telling the truth, or neither of them is.
    \part I'll have either fish or chicken, but I won’t have both fish and mashed
potatoes.
    \part  3 is a common divisor of 6, 9, and 15.
\end{parts}

\begin{solution}
    \begin{enumerate}[label=(\alph*)]
        \item J:= John is telling the truth \\
        B:= Bill is telling the truth \\
        $(J \wedge B) \vee (\neg J \wedge \neg B)$
        
        \item F:= I'll have fish \\
        C:= I'll have chicken \\
        P:= I'll have mashed potatoes \\
        $(F \vee C) \wedge \neg(F \wedge P)$
        
        \item X:= 3 is a common divisor of 6\\
        Y:= 3 is a common divisor of 9\\
        Z:= 3 is a common divisor of 15\\
        $X \wedge Y \wedge Z$
    \end{enumerate}
\end{solution}


%====================================================================================================

\subsubsection{Exercise 3}

\question Analyze the logical forms of the following statements:
\begin{parts}
    \part Alice and Bob are not both in the room.
    \part Alice and Bob are both not in the room.
    \part Either Alice or Bob is not in the room.
    \part Either Alice or Bob is not in the room.
\end{parts}

A:= Alice is in the room. \\
B:= Bob is in the room.

\begin{solution}
    \begin{enumerate}[label=(\alph*)]
        \item $A \wedge B$
        \item $\neg (A \wedge B)$
        \item $\neg A \vee \neg B$
        \item $\neg A \wedge \neg B$
    \end{enumerate}
\end{solution}

%====================================================================================================

\subsubsection{Exercise 4}
\question ...

\begin{solution}
    The two sentences that don't make sense are:
    $\neg(P, Q, \wedge R)$ and \\
    $(P \wedge Q)(P\vee R)$
\end{solution}

%====================================================================================================

\subsubsection{Exercise 5}

\question Let P:= “I will buy the pants” and \\
    S:= “I will buy the shirt.” \\
    What English sentences are represented by the following expressions?
\begin{parts}
\part $\neg(P \wedge \neg S)$
\part $\neg P \wedge \neg S$
\part $\neg P \vee \neg S$
\end{parts}
\begin{solution}
    \begin{enumerate}[label=(\alph*)]
        \item $P \wedge \neg S$:= I will buy the Pant but not the shirt. \\
                            = I will buy the Pant without the shirt. \\
                $\neg (P \wedge \neg S)$:= I won't buy the Pant without the shirt.
        \item I will neigher buy the pants nor the shirt.
        \item Either I won't buy the pants or wont't buy the shirt.
    \end{enumerate}
\end{solution}

%====================================================================================================

\subsubsection{Exercise 6}
\question Let S:= “Steve is happy” and G:= “George is happy.” \\
What English sentences are represented by the following expressions?

\begin{parts}
    \part $(S \vee G) \wedge (\neg S \vee \neg G)$
    \part $[S \vee (G \wedge \neg S)] \vee \neg G$
    \part $S \vee [G \wedge (\neg S \vee \neg G)]$
\end{parts}

\begin{solution}
The best way to solve this problem are by reducing the
expressions (\url{http://math.stackexchange.com/questions/885946/simplifying-ambiguous-statements}).
\end{solution}


%====================================================================================================

\subsubsection{Exercise 7}

\question Identify the premises and conclusions of the following deductive arguments and analyze their logical forms. Do you think the reasoning is valid?
(Although you will have only your intuition to guide you in answering
this last question, in the next section we will develop some techniques for
determining the validity of arguments.)

\begin{parts}
    \part Jane and Pete won’t both win the math prize. \\
            Pete will win either the math prize or the chemistry prize. \\
            Jane will win the math prize. \\
            Therefore, Pete will win the chemistry prize.
    \part The main course will be either beef or fish. \\
            The vegetable will be either peas or corn. \\
            We will not have both fish as a main course and corn as a vegetable. \\
            Therefore, we will not have both beef as a main course and peas as a vegetable.
    \part Either John or Bill is telling the truth. \\
            Either Sam or Bill is lying. \\
            Therefore, either John is telling the truth or Sam is lying.
    \part Either sales will go up and the boss will be happy, \\
            or expenses will go up and the boss won’t be happy. \\
            Therefore, sales and expenses will not
both go up.
\end{parts}

\begin{solution}
    \begin{enumerate}
        \item 
                J:= Jane will win the math prize \\
                P:= Pete will win the math prize \\
                C:= Pete will win the chemistry prize. \\
                
                \textbf{Premises}: \\
                $\neg (J \wedge P)$ \\
                $P \vee C$ \\
                $J$ \\
                
                \textbf{Conclusion}: C \\
                
                \textbf{Reasoning is valid}: Jane has won the math prize so Pete can't win the math prize. So he can only will the chemistry prize according to the premises.
        
        \item 
                B:= Beef will be the main course. \\
                F:= Fish will be the main course. \\
                P:= Peas will be the vegetable. \\
                C:= Corn will be the vegetable. \\
                
                \textbf{Premises}: \\
                $B \vee F$ \\
                $P \vee C$ \\
                $\neg(F \wedge C)$ \\
                
                \textbf{Conclusion}: \\
                $\neg(B \wedge P)$ \\
                
                \textbf{Reasoning seems invalid}: In a case where you cannot have both fish and corn, you will end up with Beef and Peas.
                
        \item
                J:= John is telling the truth. \\
                B:= Bill is telling the truth.\\
                S:= Sam is telling the truth.\\
                
                \textbf{Premises}:
                $J \vee B$ \\
                $\neg S \vee \neg B$ \\
                
                \textbf{Conclusion}: \\
                $J \vee \neg S$\\
                
                \textbf{Reasoning seems valid}: $J \vee B \vee \neg S \vee \neg B$ gives $J \vee \neg S$
                
        \item
                Either sales will go up and the boss will be happy, \\
                or expenses will go up and the boss won’t be happy.\\
                Therefore, sales and expenses will not both go up.
                
                S:= Sales will go up. \\
                B:= Boss will be happy. \\
                E:= Expenses will go up. \\
                
                \textbf{Premises}:
                $(S \wedge B) \vee (E \wedge \neg B)$ \\
                
                \textbf{Conclusion}:
                ¬(S ∧ E)
                
                \textbf{Reasoning is invalid}: When $S = T$, $E = T$, $B = T$, then premise is $true$ and conclusion is $false$.

        
    \end{enumerate}
\end{solution}

\end{questions}