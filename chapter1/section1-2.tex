\subsection{Section 2: Truth Tables}

\subsubsection{Exercise 1}
\question ...

\begin{solution}
    \begin{enumerate}
        \item {
            \begin{tabular}[t]{l l | l}
            p & q & $(\neg p \vee q)$ \\ \hline
            T & T & T \\
            T & F & F \\
            F & T & T \\
            F & F & T
            \end{tabular}
        }
        
        \item{
            \begin{tabular}[t]{l l | l}
            g & s & $((s \vee g) \wedge (\neg s \vee \neg g))$ \\ \hline
            T & T & T \\
            T & F & F \\
            F & T & T \\
            F & F & T
            \end{tabular}
        }
    \end{enumerate}
\end{solution}

%====================================================================================================
\subsubsection{Exercise 2}
\question ...?

\begin{solution}
\begin{enumerate}
    \item{
        \begin{tabular}[c]{l l |l}
        p & q & $\neg (p \wedge (q \vee \neg p))$ \\ \hline
        T & T & F \\
        T & F & T \\
        F & T & T \\
        F & F & T
        \end{tabular}
    }
    
    \item{
        \begin{tabular}[c]{l l l | l}
        p & q & r & $((p \vee q) \wedge (\neg p \vee r))$ \\ \hline
        T & T & T & T \\
        T & T & F & F \\
        T & F & T & T \\
        T & F & F & F \\
        F & T & T & T \\
        F & T & F & T \\
        F & F & T & F \\
        F & F & F & F
        \end{tabular}
    }
    \end{enumerate}
\end{solution}

%====================================================================================================
\subsubsection{Exercise 3}
\question ...?

\begin{parts}
    \part ...? \missing{Exercise 3, part A}
    \part Find a formula using only the connectives $\wedge$, $\vee$, and $\neg$ that is equivalent to $P + Q$.  Justify your answer with a truth table.
\end{parts}

\uncertain{I'm not sure why this is ``$P$ + $Q$''}

\begin{solution}
    \begin{enumerate}[label=(\alph*)]
        \item {
            \begin{tabular}[c]{l l | l}
            p & q & $p + q$ \\ \hline
            T & T & F \\
            T & F & T \\
            F & T & T \\
            F & F & F
            \end{tabular}
        }
        
        \item {
            \begin{tabular}[c]{l l | l}
                p & q & $((p \wedge \neg q) \vee (\neg p \wedge q))$ \\ \hline
                T & T & F \\
                T & F & T \\
                F & T & T \\
                F & F & F
            \end{tabular}
        }
    \end{enumerate}
\end{solution}

%====================================================================================================
\subsubsection{Exercise 4}

\question Find a formula using only the connectives $\wedge$ and $\neg$ that is equivalent to $P \vee Q$.  Justify your answer with a truth table.

\begin{solution}
        \begin{tabular}[c]{l l | l}
            p & q & $P \vee Q$ \\ \hline
            T & T & T \\
            T & F & T \\
            F & T & T \\
            F & F & F
        \end{tabular}
    
    From DeMorgan's Law: \\
    $\neg( P \vee Q) \equiv \neg P \wedge \neg Q$ \\
    $(P \vee Q) \equiv \neg(\neg \wedge \neg Q)$
\end{solution}



%====================================================================================================
\subsubsection{Exercise 5}

\question ...?
\begin{parts}
    \part ...
    \part ...
    \part  Find formulas using only the connective $\downarrow$ that are equivalent to $\neg P$,
$P \vee Q$, and $P \wedge Q$.
\end{parts}

\begin{solution}
    \begin{enumerate}[label=(\alph*)]
        \item {
            \begin{tabular}[c]{l l |l}
            P & Q  & P $\downarrow Q$ \\ \hline
            T & T & F \\
            T & F & F \\
            F & T & F \\
            F & F & T
            \end{tabular}
            \text{Where $P \downarrow Q$ means neither P nor Q}
        }
        
        \item {$\neg P \wedge \neg Q$}
        
        \item {
            \begin{proof}
                \begin{align*}
                    P \downarrow P \\
                    \neg (P \downarrow Q) \equiv (P \downarrow Q) \downarrow (P \downarrow Q) \\
                    \neg P \downarrow \neg Q \equiv (P \downarrow P) \downarrow (Q \downarrow Q)
                \end{align*}
            \end{proof}
        }
    \end{enumerate}
\end{solution}
%(a) p q | p ↓ q
%    T T | F
%    T F | F
%    F T | F
%    F F | T

%(b) ¬p ∧ ¬q
%(c) Find formulas using only the connective ↓ that are equivalent to ¬P,
%P ∨ Q, and P ∧ Q.
%    p ↓ p
%    ¬(p ↓ q) ≡ (p ↓ q) ↓ (p ↓ q)
%    ¬p ↓ ¬q  ≡ (p ↓ p) ↓ (q ↓ q)

%====================================================================================================
\subsubsection{Exercise 6}
\question Some mathematicians write $P\; |\; Q$ to mean "$P$ and $Q$ are not both true,” or, $\neg (P \wedge Q)$.  (This connective is called NAND, and is used in the study of circuits in computer science.)
    \begin{parts}
        \part Make the truth table for $P | Q$
        \part Using only $\wedge$, $\vee$, and $\neg$ derive a formula equivalent to $P\; |\; Q$
        \part Using only $|$ derive formulas equivalent to: $\neg P$, $P \vee Q$, and $P \wedge Q$
    \end{parts}
    
    \begin{solution}
        \begin{enumerate}[label=(\alph*)]
            \item {
                \begin{tabular}[c]{l l | l}
                    P & Q & $P\; |\; Q$ \\ \hline
                    T & T & F \\
                    T & F & T \\
                    F & T & T \\
                    F & F & T \\
                \end{tabular}
            }
            
            \item {
                \begin{itemize}
                    \item $(P\; |\; Q) \equiv \neg (P \wedge Q)$
                    \item $(P\; |\; Q) \equiv \neg P \vee \neg Q$
                \end{itemize}
            }
            
            \item {
                \begin{itemize}
                    \item $\neg P \equiv (P\; |\; P)$ 
                    \item $P \vee Q \equiv (\neg P\; |\; \neg Q) \equiv ((P\; |\; P)\; |\; (Q\; |\; Q)$
                    \item $P \wedge Q \equiv \neg(P\; |\; Q) \equiv (P\; |\; Q)\; |\; (P\; |\; Q)$
                \end{itemize}
            }
        \end{enumerate}
    \end{solution}
%====================================================================================================
\subsubsection{Exercise 7}

\question Use truth tables to determine whether or not the arguments in exercise 7 of Section 1.1 are valid.

\begin{parts}
    \part J:= Jane will win the math prize \\
            P:= Pete will win the math prize \\
            C:= Pete will win the chemistry prize\\
            Draw the truth table for: $\neg (J \wedge P) \wedge (P\vee C)\wedge J$
    \part B:= Beef will be the main course. \\
            F:= Fish will be the main course. \\
            P:= Peas will be the vegetable.\\
            C:= Corn will be the vegetable. \\
            Draw the truth table for: $(B \vee F)\wedge (P \vee C) \wedge \neg(F \wedge C)$
    \part J:= John is telling the truth \\
            B:= Bill is telling the truth\\
            S:= Sam is telling the truth. \\
            Draw the truth table for: $(J \vee B) \wedge (\neg S \vee \neg B)$
            
    \part S:= Sales will go up. \\
            B:= Boss will be happy.\\
            E:= Expenses will go up.\\
            Draw the truth table for: $(S \wedge B) \vee (E \wedge \neg B)$
\end{parts}

\begin{solution}
    \begin{enumerate}[label=(\alph*)]
        \item {
            \begin{tabular}[c]{l l l | l}
                C & J & P & $(\neg(J \wedge P) \wedge ((P \vee) \wedge J)$ \\ \hline
                T & T & T & F \\
                T & T & F & T \\
                T & F & T & F \\
                T & F & F & F \\
                F & T & T & F \\
                F & T & F & F \\
                F & F & T & F \\
                F & F & F & F \\
            \end{tabular}
            Now, the result is True when `C` and `J` are true. Hence the conclusion that Pete will win the chemistry prize is true.
        }
        
        \item {
            \begin{tabular}[c]{l l l l | l}
                B & C & F & P & $((B \vee F) \wedge ((P \vee C) \wedge \neg(F \wedge C )))$ \\ \hline
                T & T & T & T & F \\
                T & T & T & F & F \\
                T & T & F & T & T \\
                T & T & F & F & T \\
                T & F & T & T & T \\
                T & F & T & F & F \\
                T & F & F & T & T \\
                T & F & F & F & F \\
                F & T & T & T & F \\
                F & T & T & F & F \\
                F & T & F & T & F \\
                F & T & F & F & F \\
                F & F & T & T & T \\
                F & F & T & F & F \\
                F & F & F & T & F \\
                F & F & F & F & F
            \end{tabular}
            \\
            \textbf{Conclusion}: $\neg(B \wedge P)$:
            \begin{tabular}[c]{l l | l}
                B & P & $\neg(B \wedge P)$ \\ \hline
                T & T & F \\
                T & F & T \\
                F & T & T \\
                F & F & T
            \end{tabular}
            
            The argument is invalid because $\neg(B \wedge P)$ is $F$ when $B = T$ and $P = T$. But in third row when $B = T$ and $P = T$ in conjunction of premise, it is $T$. To interpret it literally, they can have Beef as main course and Peas as the vegetable, but the conclusion of the statement says otherwise.
        }
        
        \item {
            \textbf{Premises}: \\
                $J \vee B$ \\
                $\neg S \vee \neg B$ \\
            \textbf{Conclusion}: $J \vee \neg S$ \\
            \textbf{Conjunction of Premises}: $(J \vee B) \wedge (\neg S \vee \neg B)$ \\
        
            \begin{tabular}[c]{l l l | l}
            B & J & S & $((J \vee B) \wedge (\neg S \vee \neg B)$ \\ \hline
            T & T & T & F \\
            T & T & F & T \\
            T & F & T & F \\
            T & F & F & T \\
            F & T & T & T \\
            F & T & F & T \\
            F & F & T & F \\
            F & F & F & F
            \end{tabular}
            \\
            Conclusion truth table: \\
            \begin{tabular}[c]{l l | l}
                J & S & $J \vee \neg S$ \\ \hline
                T & T & T \\
                T & F & T \\
                F & T & F \\
                F & F & T
            \end{tabular}
            
            Whenever the conclusion truth table is $T$, for those parameters of $J$ and $S$, the premises truth table is also true which implies that the conclusion is valid.
        }
        \item {
            \textbf{Premises}: $(S \wedge B) \vee (E \wedge \neg B)$
            \begin{tabular}[c]{l l l | l }
                B & E & S & $((S \wedge B) \vee (E \wedge \neg B))$ \\ \hline
                T & T & T & T \\
                T & T & F & F \\
                T & F & T & T \\
                T & F & F & F \\
                F & T & T & T \\
                F & T & F & T \\
                F & F & T & F \\
                F & F & F & F
            \end{tabular}
            \\
            \textbf{Conclusion}: $(\neg S \wedge E)$ \\
            \textbf{Conclusion truth table}:
            \begin{tabular}[c]{l l | l}
                E & S & $\neg (S \wedge E)$ \\ \hline
                T & T & F \\
                T & F & T \\
                F & T & T \\
                F & F & T
            \end{tabular}
            Argument is invalid because the according to the conclusion when $E = F$ and $S = F$, the premise should be valid which is not the case.
        }
    \end{enumerate}
\end{solution}

%====================================================================================================
\subsubsection{Exercise 8}

    \question  Use truth tables to determine which of the following formulas are equivalent to each other:
    \begin{parts}
        \part $(P \wedge Q) \vee (\neg P \wedge \neg Q)$
        \part $\neg P \vee Q$
        \part $(P \vee \neg Q) \wedge (Q \vee \neg P)$
        \part $\neg(P \vee Q)$
        \part $(Q \wedge P) \vee \neg P$
    \end{parts}
    
    \begin{solution}
        \begin{enumerate}[label=(\alph*)]
            \item {
                \begin{tabular}[c]{l l | l}
                    P & Q & $((P \wedge Q) \vee (\neg P \wedge \neg Q))$ \\ \hline
                    T & T & T \\
                    T & F & F \\
                    F & T & F \\
                    F & F & T
                \end{tabular}
            }
            \item {
                \begin{tabular}[c]{l l | l}
                    P & Q & $(\neg P \vee Q)$ \\ \hline
                    T & T & T \\
                    T & F & F \\
                    F & T & T \\
                    F & F & T
                \end{tabular}
            }
            \item {
                \begin{tabular}[c]{l l | l}
                    P & Q & $((P \vee \neg Q) \wedge (Q \vee \neg P))$ \\ \hline
                    T & T & T \\
                    T & F & F \\
                    F & T & F \\
                    F & F & T
                \end{tabular}
            }
            \item {
                \begin{tabular}[c]{l l | l}
                    P & Q & $\neg(P \vee Q)$ \\ \hline
                    T & T & F \\
                    T & F & F \\
                    F & T & F \\
                    F & F & T
                \end{tabular}
            }
            \item {
                \begin{tabular}[c]{l l | l}
                    P & Q & $((Q \wedge P) \vee \neg P)$ \\ \hline
                    T & T & T \\
                    T & F & F \\
                    F & T & T \\
                    F & F & T
                \end{tabular}
            }
        \end{enumerate}
        From the truth table: (a) $\equiv$ (c) and (b) $\equiv$ (e).
    \end{solution}

%====================================================================================================
\subsubsection{Exercise 9}
\question Use truth tables to determine which of these statements are tautologies, which are contradictions, and which are neither:
    \begin{parts}
        \part $(P \vee Q) \wedge (\neg P \vee \neg Q)$
        \part $(P \vee Q) \wedge (\neg P \wedge \neg Q)$
        \part $(P \vee Q) \vee (\neg P \vee \neg Q)$
        \part $[P \wedge (Q \vee \neg R)] \vee (\neg P \vee R)$
    \end{parts}

    \begin{solution}
        \begin{enumerate}[label=(\alph*)]
            \item {
                \begin{tabular}[c]{l l | l}
                    P & Q & $((P \vee Q) \wedge (\neg P \vee \neg Q))$ \\ \hline
                    T & T & F \\
                    T & F & T \\
                    F & T & T \\
                    F & F & F
                \end{tabular}
            }
            
            \item {
                \begin{tabular}[c]{l l | l}
                    P & Q & $((P \vee Q) \wedge (\neg P \wedge \neg Q))$ \\ \hline
                    T & T & F \\
                    T & F & F \\
                    F & T & F \\
                    F & F & F
                \end{tabular}
            }
            
            \item {
                \begin{tabular}[c]{l l | l}
                    P & Q & $((P \vee Q) \vee (\neg P \vee \neg Q))$ \\ \hline
                    T & T & T \\
                    T & F & T \\
                    F & T & T \\
                    F & F & T
                \end{tabular}
            }
            
            \item {
                \begin{tabular}[c]{l l l | l}
                    P & Q & R & $((P \wedge (Q \vee \neg R)) \vee (\neg P \vee R))$ \\ \hline
                    T & T & T & T \\
                    T & T & F & T \\
                    T & F & T & T \\
                    T & F & F & T \\
                    F & T & T & T \\
                    F & T & F & T \\
                    F & F & T & T \\
                    F & F & F & T
                \end{tabular}
            }
        \end{enumerate}
        From the truth tables, (b) is \textit{contradiction}. (c) and (d) are both \textit{tautologies}. (a) is neither of them.
    \end{solution}

%====================================================================================================
\subsubsection{Exercise 10}

\question The first was checked in text, so I will skip it.  Distributive law: $P \wedge (Q \vee R) \equiv (P \wedge Q( \vee (P \wedge R)$.
    \begin{solution}
        \begin{tabular}[c]{l l l | l}
            P & Q & R & $P \wedge (Q \vee R)$ \\ \hline
            T & T & T & T \\
            T & T & F & T \\
            T & F & T & T \\
            T & F & F & F \\
            F & T & T & F \\
            F & T & F & F \\
            F & F & T & F \\
            F & F & F & F
        \end{tabular}
        
        \begin{tabular}[c]{l l l | l}
            P & Q & R & $((P \wedge Q) \vee (P \wedge R)$ \\ \hline
            T & T & T & T \\
            T & T & F & T \\
            T & F & T & T \\
            T & F & F & F \\
            F & T & T & F \\
            F & T & F & F \\
            F & F & T & F \\
            F & F & F & F
        \end{tabular}
        Yes, they are equivalent.
    \end{solution}

%====================================================================================================
\subsubsection{Exercise 11}
\question Use the laws stated in the text to find simpler formulas equivalent to these formulas. (See Examples 1.2.5 and 1.2.7.)

    \begin{parts}
        \part $\neg(\neg P \wedge \neg Q)$
        \part $(P\wedge Q) \vee (P \wedge \neg Q)$
        \part $\neg(P \wedge \neg Q) \vee (\neg P \wedge Q)$
    \end{parts}
    
    \begin{solution}
        \begin{enumerate}[label=(\alph*)]
            \item {
                \begin{proof}
                    \begin{align*}
                        \neg(\neg P \wedge \neg Q) \\
                        P \vee Q && \text{DeMorgan's law}
                    \end{align*}
                \end{proof}
            }
            
            \item {
                \begin{proof}
                    \begin{align*}
                        (P \wedge Q) \vee (P \wedge \neg Q) \\
                        ((P \wedge Q) \vee P) \wedge ((P \wedge Q) \vee \neg Q) && \text{Distributive law} \\
                        (P \vee (P \wedge Q)) \wedge (\neg Q \vee (P \wedge Q)) && \text{Associative law} \\
                        P \wedge (( \neg Q \vee P) \wedge (\neg Q \vee Q)) && \text{Absorption law \& Distributive law} \\
                        P \wedge (\neg Q \vee P) \\
                        P && \text{Absorption law}
                    \end{align*}
                \end{proof}
            }
            
            \item {
                \begin{proof}
                    \begin{align*}
                        \neg(P \wedge \neg Q) \vee (\neg P \wedge Q) \\
                        (\neg P \vee Q) \vee (\neg \wedge Q) && \text{DeMorgan's law} \\
                        (\neg P \vee Q \vee \neg P) \wedge (\neg P \vee Q \vee Q) && \text{Distributive law} \\
                        (\neg P \vee Q) \wedge (\neg P \vee Q) && \text{Idempotent law} \\
                        (\neg P \vee Q) && \text{Idempotent law}
                    \end{align*}
                \end{proof}
            }
        \end{enumerate}
    \end{solution}
%(a) ¬(¬P ∧ ¬Q)
%    => P ∨ Q    (Demorgan's law)

%(b) (P ∧ Q) ∨ (P ∧ ¬Q)
%    => ((P ∧ Q) ∨ P) ∧ ((P ∧ Q) ∨ ¬Q)  [Distributive law]
%    => (P ∨ (P ∧ Q)) ∧ (¬Q ∨ (P ∧ Q))  [Associative law]
%    => P ∧ ((¬Q ∨ P) ∧ (¬Q ∨ Q)) [Absorption law & Distributive law]
%    => P ∧ (¬Q ∨ P)
%    => P [Absorption law]

%(c) ¬(P ∧ ¬Q) ∨ (¬P ∧ Q)
%    => (¬P ∨ Q) ∨ (¬P ∧ Q) [Demorgan's law]
%    => (¬P ∨ Q ∨ ¬P) ∧ (¬P ∨ Q ∨ Q) [Distributive law]
%    => (¬P ∨ Q) ∧ (¬P ∨ Q) [Idempotent law]
%    => (¬P ∨ Q)  [Idempotent law]

%====================================================================================================
\subsubsection{Exercise 12}

\question Use the laws stated in the text to find simpler formulas equivalent to these formulas. (See Examples 1.2.5 and 1.2.7.)

\begin{parts}
    \part $\neg (\neg P \vee Q) \vee (P \wedge \neg R)$
    \part $\neg (\neg P \wedge Q) \vee (P \wedge \neg R)$
    \part $(P \wedge R) \vee ([\neg R \wedge (P \vee Q)]$
\end{parts}
    
    \begin{solution}
        \begin{enumerate}[label=(\alph*)]
            \item {
                \begin{proof}
                    \begin{align*}
                        \neg(\neg P \vee Q) \vee (P \wedge \neg R) \\
                        (\neg\neg P \wedge \neg Q) \vee (P \wedge \neg R) && \text{DeMorgan's law} \\
                        (P \wedge \neg Q) \vee (P \wedge\neg R) && \text{Double Negation law} \\
                        P \wedge (\neg Q \vee \neg R) && \text{Distributive law}
                    \end{align*}
                \end{proof}
            }
            
            \item {
                \begin{proof}
                    \begin{align*}
                        \neg(\neg P \wedge Q \vee (P \wedge \neg R) \\
                        (\P \vee \neg Q) \vee (P \wedge \neg R) && \text{DeMorgan's law} \\
                        (P \vee \neg \vee P( \wedge ( P \vee \neg \vee \neg R) && \text{Distributive law} \\
                        (P \vee \neg Q) \wedge (P \vee \neg Q \vee \neg R) && \text{Idempotent law} \\
                        (P \vee \neg Q) && \text{Absorption law}
                    \end{align*}
                \end{proof}
            }
            
            \item {
                \begin{proof}
                    \begin{align*}
                        (P \wedge R) \vee [\neg R \wedge (P \vee Q)] \\
                        (P \wedge R) \vee (\neg R \wedge P) \wedge (\neg R \wedge Q) && \text{Distributive law} \\
                        (P \wedge (R \vee \neg R)) \vee (\neg R \wedge Q) && \text{Distributive law} \\
                        P \vee (\neg R \wedge Q)
                    \end{align*}
                \end{proof}
            }
        \end{enumerate}
    \end{solution}

%(a) ¬(¬P ∨ Q) ∨ (P ∧ ¬R)
%    => (¬¬P ∧ ¬Q) ∨ (P ∧ ¬R) [Demorgan's law]
%    => (P ∧ ¬Q) ∨ (P ∧ ¬R) [Double Negation law]
%    => P ∧ (¬Q ∨ ¬R) [Distributive law]

%(b) ¬(¬P ∧ Q) ∨ (P ∧ ¬R)
%    => (P ∨ ¬Q) ∨ (P ∧ ¬R) [Demorgan's law]
%    => (P ∨ ¬Q ∨ P) ∧ (P ∨ ¬Q ∨ ¬R) [Distributive law]
%    => (P v ¬Q) ∧ (P ∨ ¬Q ∨ ¬R) [Idempotent law]
%    => (P v ¬Q) [Absorption law]

%(c) (P ∧ R) ∨ [¬R ∧ (P ∨ Q)]
%    => (P ∧ R) ∨ (¬R ∧ P) ∨ (¬R ∧ Q) [Distributive law]
%    => (P ∧ (R ∨ ¬R)) ∨ (¬R ∧ Q) [Distributive law]
%    => P ∨ (¬R ∧ Q)

%====================================================================================================
\subsubsection{Exercise 13}
    \question Use the first DeMorgan’s law and the double negation law to derive the second DeMorgan’s law. \\
    First DeMorgan's law: $\neg (P \wedge Q) \equiv \neg P \vee \neg Q$ \\
    Second DeMorgan's law: $\neg (P \vee Q) \equiv \neg P \wedge \neg Q$
    \begin{solution}
    \begin{proof}
            \begin{align*}
                \neg(P \vee Q) && \text{Left-hand side of DeMorgan's law} \\
                \neg(\neg\neg P \vee \neg\neg Q) && \text{Double Negation} \\
                \neg(\neg(\neg P \wedge \neg Q)) && \text{DeMorgan's first law} \\
                (\neg P \wedge \neg Q) && \text{Right-hand side of second DeMorgan's law}
            \end{align*}
        \end{proof}
    \end{solution}
    %¬(P ∨ Q) [LHS of second DeMorgan's law]
    %=> ¬(¬¬P ∨ ¬¬Q) [Double negation]
    %=> ¬(¬(¬P ∧ ¬Q)) [DeMorgan's first law]
    %=> (¬P ∧ ¬Q) [RHS of second DeMorgan's law]

%====================================================================================================
\subsubsection{Exercise 14}
Exercise 14
-----------
\question Use associative laws to show that:
    $[P \wedge (Q \wedge R)] \wedge S \equiv (P \wedge Q) \wedge (R wedge S)$
    Note that the associative laws say only that parentheses are unnecessary when combining three statements with $\wedge$ or $\vee$. In fact, these laws can be used to justify leaving parentheses out when more than three statements are combined. 

\begin{solution}
    \begin{align*}
        [P \wedge (Q \wedge R)] \wedge S \Rightarrow ((P \wedge Q) \wedge R) \wedge S \Rightarrow (P \wedge Q) \wedge (R \wedge S)
    \end{align*}
\end{solution}

    %[P ∧ (Q ∧ R)] ∧ S
    %=> ((P ∧ Q) ∧ R) ∧ S
    %=> (P ∧ Q) ∧ (R ∧ S)

%====================================================================================================
\subsubsection{Exercise 15}
\question ....
\begin{solution}
    $2^n$
\end{solution}

%====================================================================================================
\subsubsection{Exercise 16}
\question Let's take conjugate of all the rows where it is true and then take the disjunctions of them. Or more simpler way would be to take conjugate of where it is false and inverse them in this case:

    \begin{solution}
        \begin{align*}
            \neg(\neg P \wedge Q) \Rightarrow P \vee \neg Q
        \end{align*}
    \end{solution}
    %¬(¬P ∧ Q)
    %=> P ∨ ¬Q

%====================================================================================================
\subsubsection{Exercise 17}

\question Solve using the same strategy as above:

\begin{solution}
    \begin{align*}
        (\neg P \wedge Q) \vee (P \wedge \neg Q) \Rightarrow ((\neg P \wedge Q) \vee P) \wedge (( \neg P \wedge Q) \vee \neg Q) \Rightarrow (P \vee Q) \wedge (\neg Q \vee \neg Q)
    \end{align*}
\end{solution}
    
    %(¬P ∧ Q) ∨ (P ∧ ¬Q)
    %=> ((¬P ∧ Q) ∨ P) ∧ ((¬P ∧ Q) ∨ ¬Q)
    %=> (P ∨ Q) ∧ (¬P ∨ ¬Q)

%====================================================================================================
\subsubsection{Exercise 18}

    \question Suppose the conclusion of an argument is a tautology.
    \begin{parts}
        \part  What can you conclude about the validity of the argument?
        \part What if the conclusion is a contradiction?
        \part What if one of the premises is either a tautology or a contradiction?
    \end{parts}

    \begin{solution}
        \begin{enumerate}[label=(\alph*)]
            \item If the conclusion of an argument is a tautology, then no matter what the condition of the premise it is true for all the cases.
            
            \item Similarly if the conclusion is a contradiction, then no matter what the condition of the premises it is false for all the cases.
            
            \item Doesn't really matter as the conclusion of an argument is an tautology.
        \end{enumerate}
    \end{solution}