%%%%%%%%%%%%%%%%%%%%%%%%%%%%%%%%%%%%%%%%%
% Author: Sibi <sibi@psibi.in>
%%%%%%%%%%%%%%%%%%%%%%%%%%%%%%%%%%%%%%%%%
\documentclass{article}
\usepackage{graphicx}
\usepackage{verbatim}
\usepackage{amsmath}
\usepackage{amsfonts}
\usepackage{amssymb}
\usepackage{tabularx}
\setlength\parskip{\baselineskip}
\begin{document}
\title{Chapter 3 (Section 3.1)}
\author{Sibi}
\date{\today}
\maketitle
\newpage

\section{Problem 1}
Consider the following theorem. (This theorem was proven in the
introduction.)

Theorem. Suppose n is an integer larger than 1 and n is not prime.
Then $2^n - 1$  is not prime.

(a) Identify the hypotheses and conclusion of the theorem. Are the
hypotheses true when n = 6? What does the theorem tell you in this
instance? Is it right?

(b) What can you conclude from the theorem in the case n = 15? Check
directly that this conclusion is correct.

(c) What can you conclude from the theorem in the case n = 11?

Solution (a)

Hypotheses:
\begin{itemize}
\item n is an integer
\item $n > 1$
\item n is not prime
\end{itemize}

Conclusion:
$2^n - 1$ is not prime.

Hypotheses are true when $n = 6$. The theorem in the instance says
that $2^n - 1 = 65$ for $n=6$ is not a prime number. Yes, it is
correct.

Solution (b)

When, $n=15, 2^n - 1 = 32767/$. So, I guess that is not an prime
number. 7 is one of the factors.

Solution (c)

When, $n=11, 2^n - 1 = 2047$. So 2047 is not a prime number. And hey
23 is one of it's factor.

\section{Problem 2}
Consider the following theorem. (The theorem is correct, but we will not
ask you to prove it here.)

Theorem. Suppose that $b^2 > 4ac$. Then the quadratic equation $ax^2 +
bx + c = 0$ has exactly two real solutions.

(a) Identify the hypotheses and conclusion of the theorem.

(b) To give an instance of the theorem, you must specify values for a, b,
and c, but not x. Why?

(c) What can you conclude from the theorem in the case a = 2, b = −5,
c = 3? Check directly that this conclusion is correct.

(d) What can you conclude from the theorem in the case a = 2, b = 4, c
= 3?

Solution (a)

Hypotheses:
\begin{itemize}
\item $b^2 - 4ac$
\end{itemize}

Conclusion: $ax^2 + bx + c$ has two real solutions.

Solutions (b)

Because a, b and c are free variables.

Solution (c)

$a = 2, b = -5, c=3$, Since $25 > 24$, they must have two real
solutions. And in fact they have two real solutions: 1 and 1.5.

Solution (d)

When $a =2, b=4, c=3$, $b^2 > 4ac$ is equivalent to $16 > 24$.
Since $b^2 > 4ac$ is the premise of the implication, passing in values
that result in $16 > 24$ means those values do not make the premise
true, therefore the premise is false. In an implication a false premise
leads to an indeterminate conclusion. Therefore we cannot determine
anything from this statement, since the conclusion can be either
true or false under this condition.


\section{Problem 3}

Hypotheses:
\begin{itemize}
\item n is a natural number
\item $n > 2$
\item n is not a prime number
\end{itemize}

Conclusion: $2n + 13$ is not a prime number

Counterexample: When $n = 8$, $2n + 13 = 29$ which is a prime number.

\section{Problem 4}

Suppose a and b are real numbers. Prove that if $0 < a < b$ then $a^2 <
b^2$

%% +----------------------+----------------------------------+
%% |Givens                |Goals                             |
%% +----------------------+----------------------------------+
%% |a and b are real      | $ 0 < a < b \implies a^2 < b^2$  |
%% |numbers.              |                                  |
%% |                      |                                  |
%% +----------------------+----------------------------------+

% This LaTeX table template is generated by emacs 24.3.1
\begin{tabular}{| >{$}l<{$} | >{$}l<{$} |}
\hline
Givens & Goals \\
\hline
\text{a and b are real} & 0 < a < b \implies a^2 < b^2 \\
numbers. & \\
 & \\
\hline
\end{tabular}

%% +----------------------+----------------------------------+
%% |Givens                |Goals                             |
%% +----------------------+----------------------------------+
%% |a and b are real      |$ a^2 < b^2$                      |
%% |numbers.              |                                  |
%% |                      |                                  |
%% |$0 < a < b$           |                                  |
%% +----------------------+----------------------------------+

% This LaTeX table template is generated by emacs 24.3.1
\begin{tabular}{| >{$}l<{$} | >{$}l<{$} |}
\hline
Givens & Goals \\
\hline
\text{a and b are real} &  a^2 < b^2 \\
numbers. & \\
 & \\
0 < a < b & \\
\hline
\end{tabular}

Now, let's subtract $a$ from the inequality $a < b$ and we get $0 < b
- a$. Since $b + a > 0$, we can multiply the previous inequality
without changing the sign and we get $0 < b^2 - a^2$ or $b^2 - a^2 >
0$.

Alt Proof: Assume $0 < a < b$. Then $a^2 < ab$ and $ab < b^2$. By conjunction then
$a^2 < ab < b^2$, and therefore $a^2 < b^2$ by the transitivity axiom.

\section{Problem 5}
%% +----------------------+----------------------------------+
%% |Givens                |Goals                             |
%% +----------------------+----------------------------------+
%% |\text{a and b are real| a < b < 0 \implies a^2 > b^2     |
%% |numbers.}             |                                  |
%% |                      |                                  |
%% +----------------------+----------------------------------+

% This LaTeX table template is generated by emacs 24.3.1
\begin{tabular}{| >{$}l<{$} | >{$}l<{$} |}
\hline
Givens & Goals \\
\hline
\text{a and b are real} & a < b < 0 \implies a^2 > b^2 \\
numbers. & \\
 & \\
\hline
\end{tabular}

%% +----------------------+----------------------------------+
%% |Givens                |Goals                             |
%% +----------------------+----------------------------------+
%% |\text{a and b are real| a^2 > b^2                        |
%% |numbers.}             |                                  |
%% |                      |                                  |
%% |a < b < 0             |                                  |
%% +----------------------+----------------------------------+

% This LaTeX table template is generated by emacs 24.3.1
\begin{tabular}{| >{$}l<{$} | >{$}l<{$} |}
\hline
Givens & Goals \\
\hline
\text{a and b are real} & a^2 < b^2 \\
numbers. & \\
 & \\
a < b < 0 & \\
\hline
\end{tabular}

Let's multiply the inequality by both a and b and we get, $a^2 > ab$
and $ab > b^2$. Since they are negative the signs are changed. From
here, we can conclude that $a^2 > b^2$

Proof. Suppose $a < b < 0$. Multiplying the inequality $a < b$ by the
negative number a we can conclude that $a^2 > ab$, and similarly
multiplying the negative number b we can conclude that $ab > b^2$.
Thereforre $a^2 > ab > b^2$, so $a^2 > b^2$, as required. Thus, if $ a
< b < 0$ then $a^2 > b^2$.

\section{Problem 6}
%% +----------------------+----------------------------------+
%% |Givens                |Goals                             |
%% +----------------------+----------------------------------+
%% |a and b are real      | $ 0 < a < b \implies 1/b < 1/a$  |
%% |numbers.              |                                  |
%% |                      |                                  |
%% +----------------------+----------------------------------+

% This LaTeX table template is generated by emacs 24.3.1
\begin{tabular}{| >{$}l<{$} | >{$}l<{$} |}
\hline
Givens & Goals \\
\hline
\text{a and b are real} & 0 < a < b \implies 1/b < 1/a \\
numbers. & \\
 & \\
\hline
\end{tabular}

%% +----------------------+----------------------------------+
%% |Givens                |Goals                             |
%% +----------------------+----------------------------------+
%% |a and b are real      | $1/b < 1/a$                      |
%% |numbers.              |                                  |
%% |                      |                                  |
%% |$0 < a < b$           |                                  |
%% +----------------------+----------------------------------+
% This LaTeX table template is generated by emacs 24.3.1
\begin{tabular}{| >{$}l<{$} | >{$}l<{$} |}
\hline
Givens & Goals \\
\hline
\text{a and b are real} & 1/b < 1/a \\
numbers. & \\
 & \\
0 < a < b & \\
\hline
\end{tabular}

Now, multiply the inequality $a < b$, by $1/a$ and you get $1 < b/a$.
Now again multiply this inequality by the positive number $1/b$ and
you get $1/b < 1/a$.

Proof. Suppose $0 < a < b$. Multiplying the inequality $a < b$ by
the positive $1/a$ and we can conclude that $1 < b/a$. Again
multiplying the previous inequality by the positive number $1/b$ we
can conclude that $1/a < 1/b$. Thus, if $0 < a < b$ then $1/b < 1/a$.

\section{Problem 7}
%% +----------------------+----------------------------------+
%% |Givens                |Goals                             |
%% +----------------------+----------------------------------+
%% |a is a real number.   |  $a^3 > a \implies a^5 > a$      |
%% |                      |                                  |
%% |                      |                                  |
%% +----------------------+----------------------------------+
% This LaTeX table template is generated by emacs 24.3.1
\begin{tabular}{| >{$}l<{$} | >{$}l<{$} |}
\hline
Givens & Goals \\
\hline
a \textrm{is a real number} & a^3 > a \implies a^5 > a \\
 & \\
 & \\
\hline
\end{tabular}

%% +----------------------+----------------------------------+
%% |Givens                |Goals                             |
%% +----------------------+----------------------------------+
%% |a is a real number.   | $a^5 > a$                        |
%% |                      |                                  |
%% |$a^3 > a$             |                                  |
%% +----------------------+----------------------------------+
% This LaTeX table template is generated by emacs 24.3.1
\begin{tabular}{| >{$}l<{$} | >{$}l<{$} |}
\hline
Givens & Goals \\
\hline
a \textrm{is a real number} & a^5 > a \\
 & \\
a^3 > a & \\
\hline
\end{tabular}


Subtracting the inequality $a^3 > a$ by a, we get $a^3 - a >0$.
Multiplying it by the postive number $a^2 + 1$, we can conclude
that $a^5 - a > 0$ or $a^5 > a$.

Proof. Suppose $a^3 > a$. Subtracting the inequality $a^3 > a$ by a,
we get $a^3 - a >0$. Multiplying it by the postive number $a^2 + 1$,
we can conclude that $a^5 - a > 0$ or $a^5 > a$. Thus, if $a^3 > a$,
then $a^5 > a$.

\section{Problem 8}
%% +--------------------------------------+----------------------------------+
%% |Givens                                |Goals                             |
%% +--------------------------------------+----------------------------------+
%% |$A \setminus B \subset C \cap D$      | $x \notin D \implies x \in B$    |
%% |                                      |                                  |
%% |$x \in A$                             |                                  |
%% +--------------------------------------+----------------------------------+
% This LaTeX table template is generated by emacs 24.3.1
\begin{tabular}{| >{$}l<{$} | >{$}l<{$} |}
\hline
Givens & Goals \\
\hline
A \setminus B \subset C \cap D & x \notin D \implies x \in B \\
 & \\
x \in A & \\
 & \\
\forall x (x \in \{ y | y \in A \wedge y \notin B \} \implies x \in \{ y | y \in C \wedge y \in D \}) & \\
\textrm{        (definition of minus, subset, and intersection)} & \\
 & \\
\forall x ((x \in A \wedge x \notin B) \implies (x \in C \wedge x \in D)) & \\
\textrm{        because }x \in \{y|P(y)\} \equiv P(y) & \\
 & \\
x \notin B \textrm{ (contrapositive assumed)} & x \in D \textrm{ (contrapositive)} \\
 & \\
x \in A \wedge x \notin B \textrm{ (conjunction)} & \\
 & \\
x \in C \wedge x \in D \textrm{ (Modus Tollens)} & \\
 & \\
x \in D \textrm{ (conj. simplification, contrapos. proven)} & \\
\hline
\end{tabular}

Proof. We will prove the contrapositive. Suppose $x \notin B$. Then we
can conclude from $A \setminus B \subset C \cap D$, that $x \in D$.
Therefore, if $x \notin D$, then $x \in B$.

\section{Problem 9}
%% +----------------------+----------------------------------+
%% |Givens                |Goals                             |
%% +----------------------+----------------------------------+
%% |a and b are real      |$a < b \implies a + b/2 < b$      |
%% |numbers.              |                                  |
%% |                      |                                  |
%% |                      |                                  |
%% +----------------------+----------------------------------+
% This LaTeX table template is generated by emacs 24.3.1
\begin{tabular}{| >{$}l<{$} | >{$}l<{$} |}
\hline
Givens & Goals \\
\hline
\text{a and b are real} & a < b \implies \frac{a + b}{2} < b \\
numbers. & \\
 & \\
\ a < b & \frac{a + b}{2} < b \\
 & \\
\hline
\end{tabular}


Proof. Suppose $a < b$. Adding b to the inequality we can conclude
that $a + b < 2b$. Dividing that inequality with 2, we can conclude
that $\frac{a + b}{2} < b$. Therefore, if $a < b$, then $\frac{a + b}{2} < b$.

\section{Problem 10}

Proof. We will prove the contrapositive. Suppose $x=8$. Solving the
equation for $x=8$, we can conclude that $1/8 \neq 1/10$. Therefore,
if $\frac{\sqrt[3]{x} + 5}{ x^2 + 6} = \frac{1}{x}$, then $x \neq 8$.

\section{Problem 11}

\begin{tabular}{| >{$}l<{$} | >{$}l<{$} |}
\hline
Givens & Goals \\
\hline
\text{a and b are real} & ac \geq bd \implies c > d \\
0 < a < b & \\
d > 0 & \\
 & \\
c \leq d & ac < bd \\
\hline
\end{tabular}
 
Now, multiplying $c \leq d$ by the positve number a we can conclude
that $ac \leq ad$. Similarly multiplying $a < b$ by d we can conclude
that $ad < bd$. Since, $ac \leq ad < bd$, we can conclude that $ac <
bd$.

Proof. We will prove the contrapositive. Suppose $c \leq d$.
Multiplying $c \leq d$ by the postive number $a$ we can conclude that
$ac \leq ad$. Similarly multiplying $a < b$ by d we can conclude that
$ad < bd$. From $ac \leq ad < bd$, we can conclude that $ac < bd$.
Therefore, if $ac \geq bd$ then $c > d$.

\section{Problem 12}

\begin{tabular}{| >{$}l<{$} | >{$}l<{$} |}
\hline
Givens & Goals \\
\hline
\text{x and y are real} & y < 1 \\
3x + 2y \leq 5 & \\
x > 1 & \\
\hline
\end{tabular}

Proof. Suppose $x > 1$. From $3x + 2y \leq 5$, we can conclude that $x
\leq 5 - 2y \mathbin{/} 3$. Combining the inequality $x > 1$ and $x \leq 5 - 2y
\mathbin{/} 3$, we get $1 < x \leq 5 - 2y \mathbin{/}3$ which gives us $y < 1$. Therefore,
if $x > 1$, then $y < 1$.

Alternative proof by contrapositive:

\begin{tabular}{| >{$}l<{$} | >{$}l<{$} |}
\hline
Givens & Goals \\
\hline
\text{x and y are real} & x > 1 \implies y < 1 \\
3x + 2y \leq 5 & \\
 & \\
y \geq 1 \text{  (contrapositive assumed)} & x \leq 1 \\
2y \geq 2 & \\
3x \leq 5 - 2y & \\
3x \leq 5 - 2y \leq 3 & \\
\text{  (since $5-2=3$ is upper bound)} & \\
3x \leq 3 \text{ (transitivity axiom)}& \\
\hline
\end{tabular}

Proof. (by contrapositive) Assume $y \geq 1$, and so $2y \geq 2$. $3x + 2y \leq 5$ means 
$3x \leq 5-2y$, and $5-2y \leq 3$ since $2y$ must be at least 2. Then we have 
$3x \leq 5 - 2y \leq 3$, and by transitivity $3x \leq 3$, so $x \leq 3$. Thus the contrapositive
is proven, therefore if $x > 1$ for $3x + 2y \leq 5$ then $y$ must be less than 1.

\section{Problem 13}
\begin{tabular}{| >{$}l<{$} | >{$}l<{$} |}
\hline
Givens & Goals \\
\hline
\text{x and y are real} & [(x^2 + y = -3) \wedge (2x - y = 2)] \implies x = -1 \\
 & \\
(x^2 + y = -3) \wedge (2x - y = 2) & x = -1\\
\hline
\end{tabular}

Proof. Suppose $x^2 + y = -3$ and $2x - y = 2$. Solving for x from
both the equation, we get $x^2 + 2x + 1 =0$. From that, we can
conclude that $x = -1$. Therefore, if $x^2 + y = -3$ and $2x - y = 2$
then $x = -1$.

\section{Problem 14}

\begin{tabular}{| >{$}l<{$} | >{$}l<{$} |}
\hline
Givens & Goals \\
\hline
 & [(x > 3) \wedge (y < 2)] \implies (x^2 - 2y > 5) \\
 & \\
(x > 3) \wedge (y < 2) & x^2 - 2y > 5\\
0 < 3 < x \implies 0 < 9 < x^2 & \\
y < 2 \implies 2y < 4 & \\
2y + 5 < 4 + 5 = 9 & \\
2y + 5 < 9 < x^2 & \\
2y + 5 < x^2 & \\
\hline
\end{tabular}

Proof. Suppose $x > 3$ and $y < 2$. From theorem 3.1.2 if $0 < a < b$,
then $a^2 < b^2$. From that we can conclude that if  $0 < 3 < x$, then
$0 < 9 < x^2$. Multiplying 2 on the inequality $y < 2$, we get $2y <
4$. Adding 5 on the previous inequality we get $2y + 5 < 9$. Combining
the two inequality we can conclude that $2y + 5 < x^2$ which is $x^2 -
2y > 5$. Therefore, if $x > 3$ and $y < 2$, then $x^2 - 2y > 5$.

Alternate "intuitive" proof: (non-rigorous)

We know x is bounded below by 3 and y is bounded above by 2. If we
assume for a moment that $x = 3$ and $y = 2$, we see that $3^2 - 2(2) = 5$.
But since $x > 3$ then $x^2 > 9$, and $y < 2$ means $2y < 4$. So we have a
number larger than 9 minus a number smaller than 4, so the result
must be larger than 5. Hence if $x > 3$ and $y < 2$ then $x^2 - 2y > 5$.

\section{Problem 15}

Solution 15 (a)

The proof is wrong since $P \implies Q$ is not equivalent to $Q
\implies P$.

Solution 15 (b)

\begin{tabular}{| >{$}l<{$} | >{$}l<{$} |}
\hline
Givens & Goals \\
\hline
x \textrm{ is a real number} \wedge x \neq 4 &  (\frac{2x-5}{x-4} = 3) \implies (x = 7)  \\
 & \\
\frac{2x-5}{x-4} = 3 & x = 7 \\
\hline
\end{tabular}

Proof. Suppose $\frac{2x-5}{x-4} = 3$. Solving for x, we get $
x = 7$. Therefore, if $\frac{2x-5}{x-4} = 3$, then $x = 7$.

\section{Problem 16}

Solution 16 (a)

It misses the fact that when $x = -3$, the theorem fails.

Solution 16 (a)

Counterexample: $x = -3$
\end{document}
